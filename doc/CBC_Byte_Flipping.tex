\documentclass[12pt]{article}
\usepackage{amsmath}
\usepackage{graphicx}
\usepackage{hyperref}
\usepackage[latin1]{inputenc}

\title{CBC Byte Flipping}
\begin{document}
\maketitle

\section{Mathematics of CBC}
These are notes as to how byte flipping works. The mathematical formula are as follows.
$$P_i = D_K(C_i)\oplus C_{i-1}$$
$$C_i = E_K(P_i \oplus C_{i-1})$$
$$C_0 = IV$$

Therefore to break it we can represent it as :

$$P' = D_K(C_n) \oplus C'$$
$$C_n = E_K(P_n \oplus C_{n-1})$$

Where C' is block with the flipped bytes, $C_n$ is the block that is used to be decypted and xored with C' to form P'.

Combining the two formulas,

$$P' = D_K(E_K(P_n\oplus C_{n-1}))\oplus C'$$

Simplifying it,

$$P' = P_n\oplus C_{n-1} \oplus C'$$

Notice that we have the ciphertext when we intercept or are presented with one.This means we can know and determine what the values of $C_n$ and  $C'$ is. Here $P_n$ is what we are looking for according to the padding rule. This means all we need to find out from the equation is $P'$.Here, we will know if the padding is correct or wrong according to the pading oracle(a program that will tell us if padding is right or wrong).
\newpage

\section{Last Word Algorithm}

We have the simplified formula:
$$P' = P_n\oplus C_{n-1} \oplus C'$$
which when P' is equals to the padding, we can retrieve the plaintext since $C'$ and $C_{n-1}$ and $P_n$ are known. All we need to do is to try out value from 0 - 255 for C' (The block we want to hack). So

$$P'[K] = P_n[K]\oplus C_{n-1}[K] \oplus C'[K]$$ where K is the last byte and that $P'[K] =$ 0x1. Rearranging this formula with $P'[K]=$ 0x1 , we get, $$P_n[K] = 1 \oplus C_{n-1}[K] \oplus C'[K]$$ leaving only the plaintext as the unknown!

All we need to do is to repeat for the every subsequent letter until we have decoded a block. So lets go on to the next plaintext letter, we will need wait for $P'[K]$ to return two 0x02s as padding when we mess with $C'[K]$. This will give us:

$$P_n[K-1] = 2 \oplus C_{n-1}[K-1] \oplus C'[K-1]$$ and so on for $P_n[K-2], P_n[K-3] \cdots$


\section{References}
\begin{enumerate}
\item https://www.youtube.com/watch?v=pEdGUSGi1iM
\item https://www.youtube.com/watch?v=QhuUvrrGJbE
\item https://resources.infosecinstitute.com/cbc-byte-flipping-attack-101-approach/

\end{enumerate}

\end{document}
